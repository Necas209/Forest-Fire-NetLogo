No âmbito da Unidade Curricular de Computação Natural, foi solicitado um trabalho prático que consiste no desenvolvimento de um sistema computacional com agentes racionais usando a ferramenta NetLogo, com o propósito de simular a propagação de um incêndio florestal com base em diferentes fatores ambientais.

\section{Agentes Racionais}\label{sec:rational_agents}

Segundo Russel e Norvig~\cite{Russel2010IntelligentAgents}, um agente é qualquer coisa que perceciona o seu ambiente através de sensores e atua sobre esse ambiente através de atuadores.

Um agente racional é aquele que toma a decisão correta, sendo necessário definir o contexto da decisão, o que considera certo e o que considera errado.
Quando colocado num ambiente, o agente gera uma sequência de ações de acordo com estímulos (ou perceções)~\cite{Russel2010IntelligentAgents}.

Esta desencadeia uma sequência de estados no ambiente que, quando favorável, determina o sucesso do agente, avaliado de acordo com uma medida de desempenho.

A definição de racionalidade de um agente tem por base:
\begin{itemize}
    \item A medida de desempenho que define o critério de sucesso;
    \item O conhecimento prévio do agente sobre o ambiente;
    \item As ações que o agente pode executar;
    \item A sequência de perceções do agente até à data.
\end{itemize}

\section{Incêndios Florestais}\label{sec:forest_fires}

Nos sistemas de simulação de incêndios florestais, os agentes racionais podem desempenhar um papel crucial na prevenção e controlo destes desastres.
Podem ser programados para recolher informações sobre as condições ambientais, como temperatura, humidade e direção do vento, e tomar decisões sobre a melhor estratégia de combate ao fogo.

Além disso, estes agentes podem ser utilizados para simular o comportamento do fogo e a propagação do incêndio, permitindo que os gestores do sistema tomem decisões informadas sobre como agir.

Em Viegas~\cite{Viegas2011OverviewResearch}, o autor apresenta uma visão geral sobre a propagação de incêndios florestais e destaca a importância do estudo e gestão desses desastres devido à sua crescente incidência em anos recentes.
O autor discute os processos físicos envolvidos na propagação do fogo, especialmente em condições extremas, com ênfase na segurança e proteção de vidas humanas durante a propagação do incêndio.

Em Papadopoulos, et al.~\cite{Papadopoulos2011ASimulators}, os autores reforçam a importância da previsão da propagação de incêndios florestais e apresentam uma avaliação comparativa de diferentes modelos de simulação existentes na literatura, concluindo que o FARSITE é o mais indicado para esse tipo de previsão.
Já em Singh et al.~\cite{Singh2017ForestGIS}, os autores apresentam um estudo realizado na Floresta Taradevi, em Himachal Pradesh, Índia, no qual dados de deteção remota e SIG foram utilizados gerar a entrada necessária para a modelação de incêndios florestais recorrendo ao FARSITE\@.

Diferentes ferramentas e metodologias têm sido desenvolvidas para simplificar e melhorar estes processos, desde métodos inovadores de autómatos celulares (CA) integrados com Máquinas de Aprendizagem Extrema (ELM)~\cite{Zheng2017ForestMachine}, ferramentas ‘Web’ interativas como a FLogA (Fire Logic Animation), que permitem aos seus utilizadores simular incêndios em florestas reais com base em diferentes condições climáticas~\cite{Bogdos2013ACapabilities}, e até mesmo ambientes de simulação integrados~\cite{Finney2011ASimulation}, por exemplo, o DEVS-FIRE~\cite{Ntaimo2008DEVS-FIRE:Containment}, baseado na especificação de sistemas de eventos discretos (DEVS) e na utilização de modelos de espaço celular para simular a propagação do fogo e modelos de agentes para simular a contenção do fogo.

Os modelos de simulação de incêndios florestais são amplamente utilizados por especialistas em incêndios e combustíveis nos EUA para apoiar decisões táticas e estratégicas relacionadas com a mitigação do risco de incêndios florestais~\cite{Ager2009ApplicationAnalysis}.

A aplicação desses modelos é o resultado do desenvolvimento de um algoritmo de propagação de incêndios de tempo mínimo (MTT), desenvolvido por Finney~\cite{Ager2009ApplicationAnalysis}, que torna computacionalmente viável simular milhares de incêndios e gerar mapas de probabilidade e intensidade de queima em grandes áreas.
A pesquisa de Finney, et al.~\cite{Ager2009ApplicationAnalysis, Finney2011AStates}, demonstra uma abordagem prática para o uso de simulações de incêndios florestais em escalas muito amplas para fins de planeamento operacional e, possivelmente, pesquisa ecológica.